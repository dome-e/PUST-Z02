% !TEX encoding = cp1250
\chapter{Wzory matematyczne}
Stosujemy przecinek dziesi?tny, a~nie kropk? dziesi?tn?. Aby unikn?? dodatkowego odst?pu, stosujemy zapis \verb+\num{1,2345}+ lub \verb+\num{1.2345}+, co prowadzi do \num{1,2345}, a~nie \verb+$1,2345$+, co prowadzi do $1,2345$. Stosujemy zapis \num{1.2345e10}, a~nie $1{,}2345\times 10^{10}$. Powy?szy zapis mo?na stosowa? r�wnie? w trybie matematycznym, np. \verb+$\num{1.2345e10}$+ skompiluje si? do $\num{1.2345e10}$.

\section{Sta?e i~zmienne, indeksowanie}
Skalarne sta?e i~zmienne zapisujemy w~trybie matematycznym, np. $x$, $y$, $z$. Stosujemy indeksy dolne, np. $x_i$, g�rne, np. $x^j$, lub oba, np. $x_i^j$. Mo?na r�wnie? zastosowa? indeksy w~nawiasach, np. $y(k)$. Je?eli indeks zapisany jest czcionk? pochy??, spodziewamy si?, ?e przyjmuje on warto?? liczbow? (liczby naturalne), np. $x_i$ dla $i=1,\ldots,10$. Je?eli natomiast zastosujemy oznaczenie $x_{\mathrm{i}}$, to w�wczas indeks $\mathrm{i}$ nie przyjmuje ?adnej warto?ci, jest on integraln? cz??ci? zmiennej lub sta?ej. Dlatego oznaczaj?c horyzont sterowania stosujemy symbol $N_{\mathrm{u}}$, a~nie $N_u$, co by sugerowa?o, ?e indeks $u$ przyjmuje pewne warto?ci z zakresu liczb naturalnych. Analogicznie, sta?a czasowa ca?kowania oznaczana jest jako $T_{\mathrm{i}}$, a~nie jako $T_i$, sta?a czasowa r�?niczkowania to $T_{\mathrm{d}}$, a nie $T_d$. Sygna? warto?ci zadanej oznaczamy przez $y^{\mathrm{zad}}$, a~nie przez $y^{zad}$.

Nie nale?y stosowa? czcionki pochy?ej r�wnie? do tekst�w, kt�re uzupe?niaj? wyra?enia matematyczne, np. zamiast b??dnej postaci
\begin{equation}
y(x)=
\begin{cases}
x^2 & gdy \ x\le 0\\
x^3 & gdy \ x>0
\end{cases}
\nonumber
\end{equation}
powinno by?
\begin{equation}
y(x)=
\begin{cases}
x^2 & \textrm{gdy } x\le 0\\
x^3 & \textrm{gdy } x>0
\end{cases}
\nonumber
\end{equation}
Odst?py w trybie matematycznym wymuszamy za pomoc? instrukcji \verb+\+, \verb+\quad+, \verb+\qquad+ itd.

\section{Wektory}
Do oznaczenia wektor�w najcz??ciej stosujemy symbole pogrubione, np. $\boldsymbol{x}$, $\triangle\boldsymbol{u}(k)$. Pami?tamy, ?e w matematyce wektory zawsze s? pionowe. Wektory, kt�rych elementami s? skalary, zapisujemy wi?c jako
\begin{equation}
\triangle\boldsymbol{u}(k)=\left[\triangle u(k|k) \ \ldots \ \triangle u(k+N_{\mathrm{u}}-1|k) \right]^{\mathrm{T}}
\end{equation}
lub w~postaci
\begin{equation}
\triangle\boldsymbol{u}(k)=\left[
\begin{array}{c}
\triangle u(k|k)\\
\vdots\\
\triangle u(k+N_{\mathrm{u}}-1|k)
\end{array}
\right]
\label{w_dUk}
\end{equation}
Je?eli u?ywamy wektor�w, kt�rych elementami sk?adowymi s? inne wektory, najwygodniej zapisa? je pionowo. Np. elementami wektora (\ref{w_dUk}) s? podwektory
\begin{equation}
\triangle u(k+p|k)=\left[
\begin{array}{c}
\triangle u_1(k+p|k)\\
\vdots\\
\triangle u_{n_{\mathrm{u}}}(k+p|k)
\end{array}
\right]
\label{w_dukp}
\end{equation}
gdzie $p=1,\ldots,N_{\mathrm{u}}$. A wi?c ka?dy z~wektor�w (\ref{w_dukp}) ma d?ugo?? $n_{\mathrm{u}}$, wektor (\ref{w_dUk}) ma d?ugo?? $n_{\mathrm{u}}N_{\mathrm{u}}$.

\section{Macierze}
Do oznaczenia macierzy najcz??ciej stosujemy symbole pogrubione, np. macierz dynamiczna w~algorytmie DMC dla procesu o~jednym wej?ciu i~jednym wyj?ciu ma wymiar $N \times N_{\mathrm{u}}$ i struktur?
\begin{equation}
\boldsymbol{G}=\left[
\begin{array}
{cccc}
s_{1} & 0 & \ldots & 0\\
s_{2} & s_{1} & \ldots & 0\\
\vdots & \vdots & \ddots & \vdots\\
s_{N} & s_{N-1} & \ldots &  s_{N-N_{\mathrm{u}}+1}
\end{array}
\right]
\end{equation}
W~przypadku procesu o~$n_{\mathrm{u}}$ wej?ciach i~$n_{\mathrm{y}}$ wyj?ciach ma ona  wymiar $N\times N_{\mathrm{u}}$ i posta?
\begin{equation}
\boldsymbol{G}=\left[
\begin{array}
{cccc}
\boldsymbol{S}_{1} & \boldsymbol{0}_{n_{\mathrm{y}}\times n_{\mathrm{u}}} & \ldots & \boldsymbol{0}_{n_{\mathrm{y}}\times n_{\mathrm{u}}}\\
\boldsymbol{S}_{2} & \boldsymbol{S}_{1} & \ldots & \boldsymbol{0}_{n_{\mathrm{y}}\times n_{\mathrm{u}}}\\
\vdots & \vdots & \ddots & \vdots\\
\boldsymbol{S}_{N} & \boldsymbol{S}_{N-1} & \ldots &  \boldsymbol{S}_{N-N_{\mathrm{u}}+1}%
\end{array}
\right]
\label{w_G}
\end{equation}
gdzie ka?da z~macierzy sk?adowych ma wymiar $n_{\mathrm{y}}\times n_{\mathrm{u}}$
\begin{equation}
\boldsymbol{S}_p=\left[
\begin{array}
{ccc}
s_p^{1,1} & \ldots & s_p^{1,n_{\mathrm{u}}}\\
\vdots & \ddots & \vdots\\
s_p^{n_{\mathrm{y}},1} & \ldots & s_p^{n_{\mathrm{y}},n_{\mathrm{u}}}
\end{array}
\right]
\end{equation}
gdzie $p=1,\ldots,N$. A~wi?c macierz (\ref{w_G}) ma wymiar $n_{\mathrm{y}}N\times n_{\mathrm{u}}N_{\mathrm{u}}$.

\section{Wi?ksze wyra?enia matematyczne}
W~przypadku d?ugich wzor�w nie nale?y korzysta? z~otoczenia \verb+equation+, poniewa? wz�r taki zwykle nie~mie?ci si? na stronie o przyj?tej szeroko?ci, np.
\begin{equation}
y(k)=b_1u(k-1)+b_2u(k-2)+b_3u(k-3)+b_4u(k-4)+b_5u(k-5)-a_1y(k-1)-a_2y(k-2)-a_3y(k-3)-a_4y(k-4)-a_5y(k-5)
\end{equation}
Nale?y zastosowa? otoczenie \verb+align+, co prowadzi do wzoru
\begin{align}
y(k)&=b_1u(k-1)+b_2u(k-2)+b_3u(k-3)+b_4u(k-4)+b_5u(k-5)\nonumber\\
&\quad -a_1y(k-1)-a_2y(k-2)-a_3y(k-3)-a_4y(k-4)-a_5y(k-5)\label{w_yk}
\end{align}
Nie stosujemy otoczenia \verb+split+ z~powodu b??dnego centrowania. Numer wzoru z?o?onego z~wielu wierszy umieszczamy tylko w~ostatnim wierszu.




