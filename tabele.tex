% !TEX encoding = cp1250
\chapter{Tabele}
W~praktyce bardzo cz?sto nale?y wyr�wna? liczby wzgl?dem cyfr znacz?cych w~poszczeg�lnych kolumnach (czyli przecinek dziesi?tny ma by? we wszystkich wierszach tabeli umieszczony w~tym samym miejscu w~pionie). Do wyr�wnania liczb nale?y wykorzysta? pakiet \verb+siunitx+ (pakiety \verb+rccol+ oraz \verb+dcolumn+ maj? mniejsze mo?liwo?ci). Wszystkie przyk?ady podane w~niniejszym rozdziale wykorzystuj? pakiet \verb+siunitx+. Zwr�?my uwag?, ?e tytu?y znajduj?ce si? w~pierwszym wierszu wszystkich tabel s? wy?rodkowane (w~obr?bie kolejnych kom�rek).

Je?eli standardowa szeroko?? kolumn jest za ma?a, nale?y w~dowolnym wierszu wstawi? z~obu stron zawarto?ci kom�rki polecenia \verb+\hspace{odleg?o??}+, kt�re zapewniaj? odpowiedni? szeroko??. Modyfikacj? tak? zastosowano w~drugiej kolumnie tab.~\ref{t_wyrownanie_do_znaku_przecinek3}.

Je?eli tabela jest szersza ni? szeroko?? strony, nale?y zastosowa? otoczenie \verb+sidewaystable+ z~pakietu \verb+rotating+, co wykorzystano w~tab.~\ref{t_wyrownanie_do_znaku_przecinek4}.

W~zamieszczonych tabelkach wykorzystano polecenie \verb+\rule+ do wstawienia linii o~zerowej szeroko?ci do wierszy tabelek, kt�re s? zbyt w?skie.

\begin{table}
	[b] \caption{Por�wnanie liczby parametr�w~(LP) i~dok?adno?ci~(SSE) modeli}
	\label{t_wyrownanie_do_znaku_przecinek1}
	\centering
	\sisetup{table-format = 2.4}
	\begin{small}
		\begin{tabular}{|l|S[table-format=2]|S|S|S|}
			\hline
			\multicolumn{1}{|c|}{Model\rule{0pt}{3.5mm}} & LP & $\mathrm{SSE_{ucz}}$ & $\mathrm{SSE_{wer}}$ & $\mathrm{SSE_{test}}$ \\ \hline
			Liniowy \rule{0pt}{3.5mm}                    &  4 & 90.1815              & 70.7787              & \textemdash         \\
			Neuronowy, $K=1$                             &  7 & 10.1649              & 10.3895              & \textemdash         \\
			Neuronowy, $K=2$                             & 13 & 0.3282               & 0.3257               & \textemdash         \\
			Neuronowy, $K=3$                             & 19 & 0.2014               & 0.1827               & 0.1468                \\
			Neuronowy, $K=4$                             & 25 & 0.1987               & 0.1906               & \textemdash         \\
			Neuronowy, $K=5$                             & 31 & 0.1364               & 0.1971               & \textemdash         \\
			Neuronowy, $K=6$                             & 37 & 0.1340               & 0.2044               & \textemdash         \\ \hline
		\end{tabular}
	\end{small}
\end{table}

\begin{table}
	[b] \caption{Por�wnanie liczby parametr�w~(LP) i~dok?adno?ci~(SSE) modeli}
	\label{t_wyrownanie_do_znaku_przecinek2}
	\centering
	\sisetup{table-format = 1.4e-1}
	\begin{small}
		\begin{tabular}{|l|S[table-format=2]|S|S|S|}
			\hline
			\multicolumn{1}{|c|}{Model\rule{0pt}{3.5mm}} & LP & $\mathrm{SSE_{ucz}}$ & $\mathrm{SSE_{wer}}$ & $\mathrm{SSE_{test}}$ \\ \hline
			Liniowy\rule{0pt}{3.5mm} &  4 & 9.1815e1  & 7.7787e1  & \textemdash\\
			Neuronowy, $K=1$         &  7 & 1.1649e1  & 1.3895e1  & \textemdash\\
			Neuronowy, $K=2$         & 13 & 3.2821e-1 & 3.2568e-1 & \textemdash\\
			Neuronowy, $K=3$         & 19 & 2.0137e-1 & 1.8273e-1 & 1.4682e-1\\
			Neuronowy, $K=4$         & 25 & 1.9868e-1 & 1.9063e-1 & \textemdash\\
			Neuronowy, $K=5$         & 31 & 1.3642e-1 & 1.9712e-1 & \textemdash\\
			Neuronowy, $K=6$         & 37 & 1.3404e-1 & 2.0440e-1 & \textemdash\\ \hline
		\end{tabular}
	\end{small}
\end{table}

\begin{table}
	[b] \caption{Por�wnanie z?o?ono?ci obliczeniowej}
	\label{t_wyrownanie_do_znaku_przecinek3}
	\centering
	\sisetup{table-auto-round=true}
	\begin{small}
		\begin{tabular}{|l|S[table-format=2]|S[table-format=1.2]|S[table-format=1.2]|S[table-format=2.2]|S[table-format=2.2]|S[table-format=2.2]|S[table-format=3.2]|}
			\hline
			\multicolumn{1}{|c|}{Algorytm\rule{0pt}{3.25mm}} & \hspace{0.5cm} $N$ \hspace{0.5cm} & ${N_{\mathrm{u}}=1}$ & ${N_{\mathrm{u}}=2}$ & ${N_{\mathrm{u}}=3}$ & ${N_{\mathrm{u}}=4}$ & ${N_{\mathrm{u}}=5}$ & ${N_{\mathrm{u}}=10}$ \\ \hline
			NPL\rule{0pt}{3.5mm} & 5 & 0,3954 & 0,5326 & 0,8482 & 1,2868 & 1,9179 & \textemdash \\
			NO & 5 & 2,6129 & 5,0372 & 8,0029 & 12,6476 & 18,3668 & \textemdash \\
			NO$_{\mathrm{apr}}$\rule[-1.5mm]{0pt}{3.5mm} & \phantom{0}5 & 2,4654 & 4,3206 & 7,9801 & 15,2479 & 26,5298 & \textemdash \\ \hline
			NPL\rule{0pt}{3.5mm} & 10 & 0,6274 & 0, 7874 & 1,1366 & 1,6201 & 2,3101 & 9,1346 \\
			NO & 10 & 5,2040 & 9,0378 & 13,5571 & 19,1675 & 26,2604 & 76,5018 \\		
			NO$_{\mathrm{apr}}$\rule[-1.5mm]{0pt}{3.5mm} & 10 & 4,3828 & 7,5813 & 12,6279 & 20,0911 & 31, 7747 & 154,1544 \\ \hline
		\end{tabular}
	\end{small}
\end{table}

\begin{sidewaystable}
	[b] \caption{Por�wnanie z?o?ono?ci obliczeniowej}
	\label{t_wyrownanie_do_znaku_przecinek4}
	\centering
	\centering
	\sisetup{table-auto-round=true}
	\begin{small}
		\begin{tabular}{|l|S[table-format=2]|S[table-format=1.2]|S[table-format=1.2]|S[table-format=2.2]|S[table-format=2.2]|S[table-format=2.2]|S[table-format=3.2]|S[table-format=3.2]|S[table-format=3.2]|S[table-format=3.2]|}
			\hline
			\multicolumn{1}{|c|}{Algorytm\rule{0pt}{3.25mm}} & $N$ & ${N_{\mathrm{u}}=1}$ & ${N_{\mathrm{u}}=2}$ &
			${N_{\mathrm{u}}=3}$ &
			${N_{\mathrm{u}}=4}$ &
			${N_{\mathrm{u}}=5}$ &
			${N_{\mathrm{u}}=10}$ &
			${N_{\mathrm{u}}=15}$ &
			${N_{\mathrm{u}}=20}$ &
			${N_{\mathrm{u}}=30}$\\
			\hline
			NPL\rule{0pt}{3.5mm} & \phantom{0}5 & 0,3954 & 0,5326 & 0,8482 & 1,2868 & 1,9179 & \textemdash & \textemdash & \textemdash & \textemdash\\
			NO & \phantom{0}5 & 2,6129 & 5,0372 & 8,0029 & 12,6476 & 18,3668 & \textemdash & \textemdash & \textemdash & \textemdash\\
			NO$_{\mathrm{apr}}$\rule[-1.5mm]{0pt}{3.5mm} & \phantom{0}5 & 2,4654 &  4,3206 & 7,9801 & 15,2479 & 26,5298 & \textemdash & \textemdash & \textemdash & \textemdash\\
			\hline
		\end{tabular}
	\end{small}
\end{sidewaystable}